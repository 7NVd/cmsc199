%The Modification
%	More Theorems
%		Prove em all
%	Results
%	Data

%Conclusion

%\begin{center}
	\section{Modified Continuous Newton's Method}
%\end{center}

Neuberger's version and results are to be extended to the new version (\ref{modNM}). Similar to Neuberger's version of the continuous Newton's method, $p$ is still assumed to be a nonconstant complex polynomial.
\\

\begin{lem}
	Let $\om$ be a solution of (\ref{modNM}). Then, $p(\om(t))=p(\om_0)\ds e^{-kt}$.
\end{lem}

\textit{Proof:}
From (\ref{modNM}),
$$	p(\om)'(t) = -kp(\om(t))	$$
$\bu$ Let $\ds \frac{dp}{dt} = p(\om)'(t)$
$$\Ra \frac{dp}{dt} = -kp(\om(t))$$
$$\Ra \frac{dp}{dt} + kp(\om(t)) = 0$$
$\bu$ Integrating with the use of integrating factors \dots\\
\phantom{a}\quad\quad\quad getting the integrating factor: $\ds e^{\int k\ dt} = \ds e^{kt+c_1}$
$$\Ra \ds e^{kt+c_1}\mult\left[\frac{dp}{dt} + kp(\om(t))\right] = \ds e^{kt+c_1}\mult\left[0\right]$$
$$\Ra \ds e^{kt}\mult\left[\frac{dp}{dt} + kp(\om(t))\right] = \ds e^{kt}\mult\left[0\right]$$
$$\Ra \ds e^{kt}\mult\frac{dp}{dt} + e^{kt}\mult kp(\om(t)) = 0$$
$$\Ra \ds D_t\left[e^{kt}\mult p(\om(t))\right] = 0$$
$$\Ra \ds \int D_t\left[e^{kt}\mult p(\om(t))\right]dt = \ds\int0\ dt$$
\begin{equation}\label{forLemmaProof}
	\Ra \ds e^{kt}\mult p(\om(t)) = 0 + c_2
\end{equation}
$\bu$ Solving for $c_2$ using (\ref{forLemmaProof}) and the initial value $\om(0) = \om_0$ \dots
$$\Ra \ds e^{k(0)}\mult p(\om(0)) = c_2$$
$$\Ra p(\om_0) = c_2$$
$\bu$ Substituting this value of $c_2$ to (\ref{forLemmaProof}) \dots
$$\Ra \ds e^{kt}\mult p(\om(t)) = p(\om_0)$$
$$\Ra p(\om(t)) = p(\om_0)\ds e^{-kt}$$
\phantom{a}\hfill$\blacksquare$\\

Just as with Neuberger's, the set $\mathcal{Q}_k$ of functions $\om:\mathbb{R}\ra\mathbb{C}$ that solve (\ref{modNM}) was introduced. The next theorem simply extends Neuberger's theorems to (\ref{modNM}).

\begin{thm}
	For a given $k>0$, if $\om\in\mathcal{Q}_k$, then $u = \ds\lim_{t\to\infty}\om(t)$ exists and $p(u) = 0$.
\end{thm}
\textit{Proof: halos same sa proof ni neuberger. \textbf{Need to study}}


To further analyze the modified continuous Newton's method (\ref{modNM}), its equilibrium solutions were studied. Let $(x^*,y^*)$ be the 
fixed points of a system of differential equations
\begin{equation}	\label{SDE}
	\left\{ \begin{array}{c @{=} c}
		x'(t)	&	f(x(t),y(t))\\
		y'(t)	&	g(x(t),y(t))
	\end{array}\right. .
\end{equation}
Note that generally in complex planes, $z = x+iy$. With this, it is possible to acquire a (\ref{SDE}) where
\begin{equation}	\label{FandG}
	f(x,y) = \mathbb{R}e\left(	-\ds\frac{p(x+iy)}{p'(x+iy)}	\right),\ g(x,y) = \mathbb{I}m\left(	-\ds\frac{p(x+iy)}{p'(x+iy)}	\right) .
\end{equation}

\begin{thm}[Cauchy-Riemann Equations (\cite{CauchyRiemann})]
	Suppose that $$f(z) = f(x+iy) = u(x,y) + iv(x,y)$$ is differentiable at the point $z_0 = x_0 + iy_0$. Then the derivatives of $u$ and $v$ exist at the point $(x_0,y_0)$, and can be used to calculate the derivative at $(x_0,y_0)$. That is,
	$$f'(z_0) = u_x(x_0,y_0) + iv_x(x_0,y_0),$$ and $$f'(z_0) = v_y(x_0,y_0) + iu_y(x_0,y_0).$$
\end{thm}

\begin{thm}
	Let $z^* = x^* + iy^*$ be a zero with multiplicity $m$ of $p(z)$. Then $(x^*,y^*)$ is an asymptotically stable fixed point of the system (\ref{SDE}) with $f$ and $g$ given by (\ref{FandG}). In addition, if we denote $F(x,y)$ to the vector field $F(x,y) = (f(x,y),(g(x,y)))$, we hae that the Jacobian matrix of $F(x,y)$ at $(x^*,y^*)$ is
	\begin{equation}
		F'(x^*,y^*) = \left(\begin{array}{c c}
							-k/m & 0\\
							0 & -k/m
					\end{array}\right)
	\end{equation}
\end{thm}

\
textit{Proof:}
Since $z^*$ is a zero of $p(z)$ with multiplicity $m$, it is safe to say that $p(z) = (z-z^*)^m q(z)$ where $q(z^*)\neq0$. Let $R(z) = -kp(z)/p'(z)$.
$$\Ra R(z) = \frac{-k(z-z^*)^m q(z)}{D_x[(z-z^*)^m q(z)]}$$
$$\Ra R(z) = \frac{-k(z-z^*)^m q(z)}{m(z-z^*)^{m-1}q(z) + q'(z)(z-z^*)^m}$$
$$\Ra R(z) = \frac{(z-z^*)^m}{(z-z^*)^{m-1}}\mult\frac{-kq(z)}{mq(z) + q'(z)(z-z^*)}$$
$$\Ra R(z) = (z-z^*)\mult\frac{-kq(z)}{mq(z) + q'(z)(z-z^*)}$$
Let $R_1(z) = \ds\frac{-kq(z)}{mq(z) + q'(z)(z-z^*)}$,
$$R(z) = (z-z^*)R_1(z)$$
From this,
$$R'(z) = R_1(z) + (z-z^*)R_1\:\!'(z)$$
$$\Ra R'(z^*) = R_1(z^*) + (z^*-z^*)R_1\:\!'(z^*) = R_1(z^*)$$
$$\Ra R'(z^*) = \frac{-kq(z^*)}{mq(z^*) + (z^*-z^*)q'(z)}$$
$$\Ra R'(z^*) = -\frac{k}{m}$$
$\bu$ By the Cauchy-Riemann equations,
$$-\frac{k}{m} = R'(x^*+iy^*) = f_x(x^*,y^*) + ig_x(x^*,y^*) = g_y(x^*,y^*) - if_y(x^*,y^*).$$
$\bu$ By equating the above equation by the real and imaginary parts, $-k/m = f_x(x^*,y^*) = g_y(x^*,y^*)$ and $0 = g_x(x^*,y^*) = f_y(x^*,y^*)$. The Jaconian matrix then of $F(x,y)$ will be
$$F'(x^*,y^*) = \left(\begin{array}{c c}	f_x(x^*,y^*) & f_y(x^*,y^*)\\	g_x(x^*,y^*) & g_y(x^*,y^*)	\end{array}\right)$$
$$\Ra F'(x^*,y^*) = \left(\begin{array}{c c}	-\ds\frac{k}{m} & 0\\	0 & -\ds\frac{k}{m}	\end{array}\right)$$
$\bu$ Computing for the eigenvalues \dots
$$\Ra F'(x^*,y^*) = \left(\begin{array}{c c}	-\ds\frac{k}{m}-\lambda & 0\\	0 & -\ds\frac{k}{m}-\lambda	\end{array}\right)$$
$$\Ra \left(-\ds\frac{k}{m}-\lambda\right)\mult\left(-\ds\frac{k}{m}-\lambda \right)-0 = 0$$
$$\Ra \left(\ds\frac{k}{m}+\lambda\right)^2 = 0$$
$$\Ra \lambda = -\frac{k}{m},\quad\textrm{multiplicity 2}$$
$\bu$ Since $\lambda_1 = \lambda_2 < 0$, $(x^*,y^*)$ is an asymptotically stable fixed point of (\ref{SDE}).
\phantom{a}\hfill$\blacksquare$
\\

\textit{They had a remark I do not understand what happened. Basta bumabagal yung convergence kapag lumalaki si $m$ pero bumibilis kapag lumalaki si $k$.}\\

\subsection{Numerical examples}

In this part, he theoretical results will be illustrated.